% !TeX spellcheck = de_DE

\chapter*{Kurzfassung}

Diese Bachelorarbeit widmet sich der Untersuchung des Einsatzes optischer Marker zur Verbesserung von Inventurprozessen in automatischen Lagerzellen. 
Die Arbeit stützt sich auf Erkenntnisse aus den Arbeiten von \cite[Hübler (2019)]{Hübler2019} und \cite[Kistner (2017)]{LarsKistner2017}
und nutzt zwei Kameras zur Marker-Erkennung: Eine Übersichtskamera in der oberen Ecke der Lagerzelle und eine Kamera, die am Greifer angebracht ist.

Die Implementierung der Inventurprozesse erfolgt in der Warehouse Management Software, wobei Python 3.11 im Backend und das Qt Framework mit PySide6 Bindings im Frontend verwendet werden. 
Die Bildverarbeitung und Marker-Erkennung erfolgt mithilfe der Open-Source-Bibliothek OpenCV, und die perspektivische 
Entzerrung wird mit dem Paket scikit-image durchgeführt.

Im Vorfeld dieser Arbeit wurden umfangreiche Vorarbeiten im Rahmen meiner \cite[Semesterarbeit]{Semesterarbeit} geleistet, 
darunter die Erarbeitung einer Softwarearchitektur für die bestehende Lagerverwaltungssoftware Lagerverwaltung 3.0 (Rodriguez, 2019) 
sowie Überlegungen zur Auswahl der Kameras. Anlass zur Neuimplementierung der Lagerverwaltungssoftware ist der Wunsch einer 
einheitlichen Programmiersprache in der Modellfabrik $\mu$Plant und der Wechsel des Betriebssystems von Windows 7 auf Windows 10.
Die ursprüngliche Implementierung in C$\#$ funktioniert nicht mehr fehlerfrei unter Windows 10.
Außerdem baut die alte Implementierung auf dem Protokoll Modbus auf, das in der $\mu$Plant durch den Industriestandard OPC UA ersetzt wird.

Die Arbeit beinhaltet viele Aspekte der modernen Softwareentwicklung, einschließlich der Implementierung von Plugins, 
Integration von Legacy-Schnittstellen, Portierung der M2M-Kommunikation von Modbus nach OPC UA und eine Forschungskomponente zur arUco Marker-Erkennung. 

Abschließend werden eigene Überlegungen zur grundsätzlichen Erkennung der Behälter präsentiert, 
die als Anregung für zukünftige Forschungsarbeiten dienen sollen.

\ifthenelse{\equal{\Typ}{MSc}}{
\vfill
\section*{Summary}
\begin{otherlanguage}{english}
Awaken the reader's interest here!

Why should he read this - exactly this - work?
\end{otherlanguage}

\selectlanguage{ngerman}
}{}

%%% Local Variables:
%%% mode: latex
%%% TeX-master: "../MRT-Bericht-2020"
%%% End:


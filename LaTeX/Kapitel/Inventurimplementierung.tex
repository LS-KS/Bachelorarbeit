\chapter{Implementierung des Inventurprozesses}

Von den konzeptionierten Inventurverfahren aus Kapitel \ref{Inventurkonzepte} ist nur die Inventur auf Anforderung teilweise implementiert. 
Zum Einen liegt dies an einer fehlenden Leitung, die bis zur Abgabe meiner Arbeit nicht eingetroffen ist,
zum Anderen an der Kommunikation zwsichen PC und Robotersteuerung. 

Implementiert ist die FUnktion der Übersichtskamera. 
Über das Hauptfenster kann in der Menüleiste das Kameraplugin aufgerufen werden. Mit der Taste \glq Übersichtskamera\grq{} wird über die Übersichtskamera ein Bild aufgenommen und verarbeitet. 
Konkret wird dabei ein Signal vom Hauptthread an die Instanz des Moduls \verb|stocktaker| gesendet. 
Dieses ruft die Methode \verb|evaluate_storage_cam| auf. Die Ergebnisse sind verarbeitete Bildsegmente und IDs der erkannten Becher.
Außerdem werden Paletten erkannt. 
Die Ergebnisse der Methoden werden in das Datenmodell \verb|stockmodel| geschrieben. 
Dieses Datenmodell erbt von der Qt Klasse \verb|QAbstractTableModel|, wodurch die Daten aus dem Modell in der GUI angezeigt werden können.
Mit jeder Aktualiserung des Datenmodells wird das GUI automatisch aktualisiert.

Das GUI besteht aus einer Tabelle in dessen Zeilen die Lagerreihen dargestellt sind.
Dazu gibt es eine horizontale Listenansicht in dem der Kommissioniertisch gerendert wird und eine Tastenleiste im unteren Fensterbereich. 
Für jeden Palettenplatz existiert ein Delegate, das in der Datei \verb|StocktakingDelegate.qml| beschrieben ist.
Das Delegate ist mittig vertikal geteilt. Links wird der Status vvor der Inventur angezeigt, rechts der Status nach der Inventur.
Stimmen die Daten zu Palette, BecherA und BecherB überein, so wird dies mit einem grünen Haken angezeigt.
Ist die Inventur für diesen Platz abgeschlossen, wird der rote Rand des Delegates grün. 
Ist keine Palette vorhanden, ist das für die Visualisierung genutzte Rechteck transparent. 
Gleiches gilt für die Rechtecke der Becher, wenn eine leere oder teilweise leere Palette vorhanden ist.

Jedes Delegate hat zudem eine Taste \glq Details\grq{}. Mit Klick auf die Taste wird der \\
\verb|StocktakingDetail.qml| geöffnet.
Hier werden die zugehörigen Bilder angezeigt.
Oben Mittig ist der Bildausschnitt für die Palettenerkennung, darunter die Sektionen für Becher A und Becher B. 
Ist kein Bild vorhanden, wird ein graues Platzhalterbild angezeigt.
Über den Becherbildern ist die jeweils erkannte ID des AruCo Markers angezeigt.

Bilder können im Qt Framework dynamisch nur über eine Instanz der Klasse \verb|QQuickImageProvider| an das GUI übergeben werden.
In der QML Datei muss ein Image Type angelegt werden, dessen source property mit einer URL auf den QQuickImageProvider und das Bild verweist.
z.B.: \\ \verb|source: "image://stocktaking/Bild.png"|.\\
Jedes Mal, wenn sich die URL des Types ändert, wird der QQuickImageProvider das entsprechende Bild laden. 
Dies funktioniert nicht, wenn die URL gleich bleibt.
Der Ansatz, die URL bei Bedarf neu zu setzen, aber nicht zu ändern, wird kein neues Bild laden.
Dies geschieht aus Performance Gründen. 
Um Ein Bild laden zu können muss in den C++ Backend die virtuelle Funktion \verb|load()| aufgerufen werden. 
Die Methode selbst ist protected und es gibt keine andere bekannte Möglichkeit als den source String zu ändern.
JavaScript ist wie Python dynamisch typisiert und erlaubt es einen  boolschen Wert an einen String anzufügen. 
Die boolsche Variable wird dann als \glq 0\grq{} oder \glq 1\grq{} in den String geschrieben.
In vielen Beispielen findet sich die Lösung, bei jedem Aufruf die boolsche Variable zu invertieren und so den String zu ändern.
Das Ändern der URL triggert dann das Laden des Bildes.

Ich konnte jedoch beobachten, dass mit den neuen Versionen von Qt dieses Verhalten nur noch exakt drei Mal funktioniert. 
Die Ursache dafür konnte ich nicht herausfinden.
Die Lösung scheint aber zu sein, dass an die URL ein Integerwert angehängt wird, der bei jedem Aufruf inkrementiert wird.

Mit Klick auf die Taste \glq Details\grq{} wird ein neues QML Type \verb|StocktakingDetail.qml| erstellt und geladen. 
Sobald der Dialog fertig geladen ist, ändert eine JavaScript Funktion die URL der Image Types welches in dem Python Modul \verb|stocktaker| 
die Methode \verb|requestImage| aufruft. Dabei wird die URL des Bilds übergeben. 
Die URL enthält Informationen zum angeforderten Bildsegement, die mit einfachen Methoden extrahiert werden.
Das Bild, das zurückgegeben wird, muss ein QVariant Objekt sein - naheliegender Weise ein QImage.
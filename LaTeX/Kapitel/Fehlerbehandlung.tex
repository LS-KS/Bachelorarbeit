\chapter{Fehlerbehandlung}\label{Fehlerbehandlung}

In diesem Kapitel sollen mögliche, erwartbare Fehler behandelt werden. 
Fehler können systematischer Natur, z.B. als Fehler im Quellcode, oder sporadischer Natur, z.B. als Eingabefehler, sein.
Fehler haben unterschiedliche Auswirkungen auf das System, daher kann nicht jedem Fehler mit ähnlichen Maßnahmen begegnet werden.

\section{Fehleridentifikation}

\subsection{Fehlerquellen im Datenmodell}

Das Datenmodell bestimmt den den Ablauf des inhärenten Datenfluss über die Beziehungen zwischen den Datenklassen untereinander.
Der \verb|invController| stößt den Datenfluss an einner Stelle an, das Datenmodell reagiert daraushin mit ggf. weiteren Änderungen um 
konsistent zu bleiben. 
Da selbstreferenzierende Datenmodell auf Paletten, Becher und Produkte zusammen mit ihren Lagerorten beschränkt ist,
lassen sich die Fehlerquellen auf die folgenden Punkte reduzieren:
\begin{enumerate}
    \item Laden eines fehlerhaften Lagerbestands aus dem \verb|StorageData.yaml|
    \item Fehlerhafte manuelle Überschreibung des Lagerbestands (Paletten, Becher, Produkt)
    \item Änderungen am Datenmodell im Rahmen von Wartungs- oder Erweiterungsarbeiten
\end{enumerate} 

Die Datei \verb|StorageData.yaml| wird jedes mal überschrieben, wenn eine Änderung am Datenmodell erfolgt.
Der Syntax des verwendeten YAML-Formats ist sehr einfach und lässt sich leicht überprüfen. 
Dennoch ist es wahrscheinlich, dass bei manuellen Änderungen Fehler auftreten, die ein Einlesen ganz oder teilweise verhindern.

Die Software bietet die Möglichkeit über das GUI den Lagerbestand manuell zu überschreiben. 
Der \verb|invController| übernimmt dabei im Backend die angeforderte Änderung und aktualisiert das Datenmodell.
Dies führt dazu, dass weitere Änderungen am Datenmodell ausgelöst werden, um die Konsistenz zu wahren.\\
Beispiele:\\

\begin{enumerate}
    \item Ein Benutzer möchte im Lagerplatz L5a einen Becher mit der ID 1234 eingeben. 
    Er übersieht jedoch, dass ein anderer Becher mit der ID 1234 bereits im Lagerplatz L13b steht.
    Der \verb|invController| interpretiert die manuelle Änderung NICHT als eine Verschiebung des Bechers von L13b nach L5a, 
    da die Speicheradressen der beiden Becherobjekte nicht gleich sind. 
    Es handelt sich nicht um das gleiche Objekt im Sinne des Datenmodells.
    Das Endergebnis ist ein Lagerbestand in dem zwei Becher mit der ID 1234 existieren.
    \item Ein Benutzer löscht den Becher in einer Palette. 
    Der \verb|invController| arbeitet folgende Schritte ab: 
    \begin{enumerate}
        \item Ermittle die Palette in dem der Becher stand
        \item Ermittle das Produkt, welches der Becher enthält
        \item Entferne die Beziehung des Bechers zum Produkt
        \item Entferne die Beziehung des Bechers zur Palette
        \item Lösche das Becherobjekt
    \end{enumerate}
\end{enumerate}

Wenn Änderungen am Datenmodell vorgenommen werden, kann es zu Fehlern kommen, wenn die Änderungen nicht konsistent sind.
Selbst wenn die Änderungen in bestem Wissen über die nötigen Schritte vorgenommen werden, können Schritte übersehen werden, 
die dazu führen, dass das Datenmodell inkonsistent wird.

\subsection{Fehlerquellen im Viewmodel}

Viemodel liefern Teile des Datenmodells an die GUIs. Werden als eigene Klasse implementiert, die von einer Qt-Klasse erben.
Die vererbenden Klassen liefern einen Datz Methoden, von denen einige verbindlich überschrieben werden müssen, andere sind optional. 
Bei den Methoden sind Ein- und Ausgabeparameter festgelegt, die Implementierung ist jedoch frei und ergibt sich aus dem Datenmodell,
welches das Viewmodel zur Verfügung gestellt bekommt.
Geschehen hier Fehler, führt dies dazu, dass das GUI keine Daten erhält, falsche Daten anzeigt oder die Anwendung abstürzt.
Bei Implementierungen der Klasse \verb|QSortFilterProxyModel| besteht zudem schnell die Gefahr, dass Sortier- oder Filteralgorithmen
falsch oder gar nicht ausgeführt werden, sodass auch hier Gründe für fehlende Daten im GUI entstehen können.
Fehler dieser Art sind als Entwickler im Allgemeinen frustrierend, weil es kein klares Schema zu Fehlerfindung gibt.

\subsection{Fehlerquellen in Controller-Klassen}

Controller erhalten Signale vom GUI und führen daraufhin Aktionen aus. 
Diese können Änderungen am Datenmodell auslösen oder Funktionen eines Services aufrufen. 
In den allermeisten Fällen geschieht Beides.
Fehler in der Implentierung eines Controllers führen zu falscher Funktionalität oder Abstürzen der Anwendung.

\subsection{Fehlerquellen in Service-Klassen}

Serviceklassen werden mit Programmstart instanziert und Controller Klassen als Property gesetzt. 
Ihre Methoden werden von Controllern aufgerufen und führen unterschiedliche Funktionen aus. 
Fehler in der Implementierung führen also zu einer großen Bandbreite an möglichen Fehlern. 
Eine Weitere Fehlerquelle in Serviceklassen sind Ausnahmefehler, die durch den Aufruf einer Servicemethode geworfen werden können, 
wenn ein Fehler auftritt. 
Dadurch, dass Service Klassen in der Regel die Nutzung einer importierte Bibliothek mimplementieren, deren Nutzung oft Netzwerkfunktionen beinhaltet,
wird es zunehmend schwerer alle möglichen Fehlerquellen zu indetifizieren und abzufangen. 


\subsection{Fehlerquellen in QML Dateien}

Der Syntax von QML Dateien ist einfach gestaltet und Syntaxfehler werden entweder von der IDE QT Creator oder zur Laufzeit vom QML-Interpreter
erkannt und in deer Konsole ausgegeben. Syntaxcontrolle ist in nennenswertem Umfang nur im Qt Creator möglich. Fehler die auf einem 
fehlerhaften Syntax beruhen sind also vermehrt bei der Nutzung anderer IDE's zu erwarten. 

Eine häufigere Fehlerquelle sind die Eigenschaften der verschiedenen QML Types selbst. \\
Erfahrungen dieser Arbeit sind, dass Layout-Funktionen sich nicht immer intuitiv verhalten und oft mehrmals getestet werden müssen. 

Das in Qt verbreitete Signal / Slot - Prinzip führt ist ebenfalls eine häufige Fehlerquelle. 
Es muss immer sichergestellt werden, dass die übergebenen Datentypen korrekt sind, wenn die Kommunikationsrichtung 
Python zu QML ist. Der Datentyp \verb|None| existiert im javascript Syntax nicht, kann von der \verb|QMLApplicationEngine|
nicht interpretiert werden und führt zu einer Fehlermeldung in der Konsole. 
Zudem wird die gewünschte Funktion in der Regel nicht ausgeführt.
Damit Slots in einem Python Objekt ausgeführt werden können, wie in \ref{qmlUseContextProp} gezeigt, muss es nach der instanzierung als \verb|ContextProperty| der 
\verb|QMLApplicationEngine| gesetzt werden.
\ref{qmlContextProp} zeigt die Zuweisung eines Python Objekts als \verb|ContextProperty| in der Datei \verb|main.py|.

\clearpage
\lstinputlisting[language=Python, caption=QMLApplicationEngine Contextproperty Zuweisung, label=qmlContextProp]{Listings/rootContextExample.py}

\lstinputlisting[language=Python, caption=QMLApplicationEngine Contextproperty Benutzung, label=qmlUseContextProp]{Listings/useRootContextQML.qml}

Wird das Gui über den Qt Designer des Qt Creators erstellt, werden die QML Dateien automatisch generiert.
Dies führt bei einfachen GUI's zu schnellen und guten Ergebnissen. 
Einige QML Types , z.B. \verb|StackLayout|, sind im Designer jedoch nicht verfügbar. Bei der Benutzung wird im Designer eine Fehlermeldung angezeigt. 
Diese Fälle müssen \glq von Hand\grq anhand der Referenzdokumentation implementiert werden.
Passieren in dem Fall Fehler, wird oft die Anwendung oder das PlugIn nicht gestartet.

\section{Fehlerbehandlung und Fehlerprävention}

Für Syntaxfehler in Python sind die meisten IDE's mit Syntaxhighlighting und Syntaxcheck ausgestattet. 
Die Fehlermeldungen des Python-Interpreters sind in der Regel sehr aussagekräftig und helfen bei der entsprechenden Fehlerbehebung.

Syntaxfehler in QML Dateien werden vom QML-Interpreter erkannt und (meistens) in der Konsole ausgegeben.
Die Fehlermeldung des QML-Interpreters ist nicht so ganz aussagekräftig wie die des Python-Interpreters.
Verbreitet wird die Fehlermeldung an Anderer Stelle ausgegeben als der Fehler auftritt. 
Wenn javascript Syntax verwendet wird, müssen Fehler in der Regel über die Funktionsprobe ermittelt werden. 
Im Zeitverlauf der Arbeit habe ich es nicht in einem zufriedenstellendem Maß erreicht QML Dateien debuggen zu können. 
Die Entwicklung erfolgte an vielen Stellen nach dem Prinzip \grqq Trial and Error\grqq.

Teile der Anwendung, von denen erwartet werden kann, dass Fehler auftreten, solche wie
\begin{itemize}
    \item Das Laden von Daten aus einer Datei
    \item Das Aufrufen einer asynchronen Funktion
    \item Das Aufrufen einer Netzwerkfunktion
\end{itemize}

Sollte mit \verb|try|/\verb|except| behandelt werden. Es gehört zu gutem Ton in der Softwareentwicklung Exceptions einzeln abzufangen und zu behandeln. 
Jede Exception wird über die Klasse \verb|EventlogService| protokkolliert und im GUI ausgegeben.
Auf diese Weise können Fehler schnell erkannt, bewertet und behoben werden.

\subsection{Benutzereingaben und Einstellungen}

In diesem KOntext sind Benutzereinagben als von Benutzern und Administratoren eingegebene Programmeinstellungen zu verstehen. 

IP's sind dabei ein Sonderfall. Sie sind eine Verkettung von 4 Zahlen, die jeweils zwischen 0 und 255 liegen und durch einen Punkt getrennt sind.
Im Einstellungsdialog \verb|PreferenceDialog.qml| und im \verb|RfidDelegate.qml| werden eingegebene IP's mit einem 
Regular Expression Pattern überprüft. \\
Dazu wird der QML Typ \verb|RegularExpressionValidator| benutzt. 
In \ref{qmlIpVal} wird die Verwendung anhand des Textfelds aus dem \verb|RfidDelegate.qml| gezeigt.

Analog erfolgt eine Prüfung in Python, wenn IP Adressen aus der YAML Datei in die Einstellungen geladen werden.
Hier lässt sich der eingelesene String mit einfachem Python Syntax validieren. 
Der Die Methode \ref{pyIpVal} stammt aus der \verb|Preference| Klasse des Datenmodells und zeigt die Validierung in Python.

\lstinputlisting[language=Python, caption=Python RegularExpressionValidator für die Eingabe von IP's, label=pyIpVal]{Listings/regExpIp.py}

\lstinputlisting[language=Python, caption=QML RegularExpressionValidator für die Eingabe von IP's, label=qmlIpVal]{Listings/regExpIp.qml}

Der Port wird in diesen Fällen mit dem QML Type \verb|IntValidator| überprüft bzw. in Python mit dem zulässigen Wertebereich
$1024< Port < 65536$ verglichen.

URL's werden in der Anwendung nicht überprüft. 
Die erlaubten Formate sind zu unterschiedlich und die erlaubten Wertebereiche der Segmente zu groß.
Die verwendeten URL's werden z.B. im \verb|OpcuaService| verwendet. Dieser Dienst wirft eine Exception, wenn die URL nicht gültig ist. 
Die Exception wird abgefangen und im GUI protokolliert. 

\subsection{Fehler im Datenmodell}

Fehlerhafte Daten im Datenmodell können nur dann verhindert werden, wenn eine Änderung des Lagerbestands über das GUI oder über die 
Programmautomatik erfolgt.
Die manuelle Eingabe von Daten in die entsprechende .YAML Datei kann zu Fehlern führen, die nicht abgefangen werden.
Bei der Überschreibung des Lagerbestands über das GUI wird zu dem von dem Benutzer eine ausreichende Kompetenz erwartet und das Wissen, 
dass durch einen falschen Lagerbestand im Lager Schäden entstehen können. 

Die GUI Elemente \verb|StorageDialog|, \verb|GripperDialog|, \verb|WorkBenchDialog| und \\ \verb|TurtleDialog| implementieren 
die entsprechenden Funktionen.
Die Eingabe der Lagerorte und ID's erfolgt dabei über Dropdown-Menüs, deren Wertebereich systematisch und dynamisch überprüft und 
angepasst wird. 

Um in der Entwicklung die Konsistenz des Datenmodells zu gewährleisten zu können wurden zwei Tests geschrieben, die das Python Paket 
\verb|pytest| nutzen. In dem Test werden Objekte des Datenmodells erzeugt und entsprechend ihrer Intention zugewiesen. 
anschließend wird anhand von Assertions überprüft ob alle Methoden das gewünschte Verhalten zeigen. 

Ein weiterer Test wurde für den Controller des Inventars geschrieben. Mit ihm wird getestet ob die Selbstreferenzierung auch von außen funktioniert, 
sodass der Entwickler den Controller auf einfache Weise nutzen kann. 

\subsection{Fehler in Kommissionsaufträgen}

Im Automatikbetrieb erhält die Anwendung Kommissionsaufträge über die OPC UA Schnittstelle.
Ein Benutzer kann zudem manuell Kommissionsaufträge erstellen, wenn er das dafür vorgesehene PlugIn benutzt.
Kommissionsaufträge können auf mehrere Arten ungültig sein: 
\begin{itemize}
    \item Das notwendige Produkt ist nicht im Lager
    \item Im Startort steht keine Palette
    \item Im Startort steht der falsche oder kein Becher
    \item Im Zielort steht bereits eine Palette (Kollision)
    \item Im Zielort steht bereits ein Becher (Kollision)
\end{itemize}
Um diese Fälle zu verhindern, wird mit jedem Eingang einer neuen Kommission eine Validierung durchgeführt.
Dazu wird ein Abbild des aktuellen Inventars erzeugt und dann der Reihenfolge entsprechend jede anstehende Kommission auf dem Abbild durchgeführt. 
Entsteht währenddessen ein Fehler, wird eine entsprechende Warnung über den \verb|EventlogService| emittiert und die 
Automatik abgebrochen.
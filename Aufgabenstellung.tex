%! Author = lennartschink
%! Date = 29.05.23

% Preamble
\documentclass[11pt]{scrartcl}
\title{Software-Migration für die robotergestützte Lagerverwaltung mit kameragestützten Validierungsprozessen}
\subtitle{Erstellung einer integrierten Python-Anwendung mit kameragestützen Validierungsprozessen in der Industrie 4.0-Plattform Modellfabrik $\mu$Plant}
\author{Lennart Schink}
\date{\today}
% Packages
\usepackage{amsmath}
\setcounter{tocdepth}{3}
% Document
\begin{document}
    \maketitle
    \tableofcontents
    \newpage

    \section{Einleitung}

    \subsection{Einführung und Motivation}

    \subsection{Zielsetzung}
    Erstellen einer Software die drei bestehende Programme integriert. Die ganze Software wird nach Python emigriert.
    Es wird eine Software implementiert, die
    \begin{itemize}
    \item zwei Kamweras verwaltet
    \begin{itemize}
        \item  eine hochauflösende: Erfasst den gesamten Raum, Bild wird anhand markern aufgeteilt, und versucht in den Segmenten die marker der Becher zu erkennen
        \item  eine weitere Kamera wird auf dem Arm montiert. Bild dienst zur lokalen Erkennung der Marker vor/ im Greifer.
    \end{itemize}
    \item Auf Aufforderung des Benutzers oder in festgelegten Abständen oder aufgrund der Türöffnung eine automatisierte Inventur durchführt:
    \begin{itemize}
        \item  Versuch die Inventur über die hochauflösende Kamera durchzuführen.
        \item  Zur Verifizierung / bei Nichterkennung oder auf Aufforderung:
    \end{itemize}
    \item -Fahrt des Greifers vor dem Regal in einer Haltung die das bestmögliche Erkennen der Becher ermöglicht
    \item -Paletten werden auf die Werkbank gestellt und dort Marker auf den Bechern erkannt.
    \end{itemize}

    \newpage
    \section{Theoretischer Hintergrund}

    \subsection{Die Programmierpsrache Python}
    -kurzes Vorstellen der Sprache
    -verwendete Bibliotheken außer nachfolgende untertitel

    \subsection{GUI Programmierung mit PySide6 und Qt QuickQML}
    - Datenmodelle
    - Signal/Slot - Prinzip

    \subsection{OPC UA}
    - Konzept
    - Code

    \subsection{TCP/IP mittels websocket Bibliothek}
    - Konzept
    - Code

    \subsection{RFID}
    - Konzept
    - Code
    - Erkennungsraten?

    \newpage
    \section{Stand der Forschung}
    \subsection{Merkmalserkennung}
    \subsection{arUco Marker}
    \subsection{Neuronale Netze ? }

    \newpage
    \section{Methodik}
    \subsection{Architektur der Python-Anwendung}
    \begin{itemize}
        \item Bezugnahme auf Semesterarbeit
        \item Vorstellen der neuen Softwarearchitektur
    \end{itemize}

    \subsection{Implementierung der Lagerverwaltung}
    \begin{itemize}
        \item Besonderheiten in der Implementierung
        \item Controller <-> DatenModell <-> GUI-Element Zuordnung? oder Klassen/Objektdiagramm?
    \end{itemize}

    \subsection{Implementierung der Merkmalserkennung}

    \subsubsection{Bildgewinnung}
    \begin{itemize}
        \item Datengenerierung von den Bildsensoren
        \item Preprozessing und Bereitstellung an GUI
    \end{itemize}

    \subsubsection{Alghorithmus zur Merkmalserkennung}
    \begin{itemize}
        \item openCV cv2 arUco Generierung
        \item openCV cv2 arUco Erkennung
        \item angewandte Bildverarbeitungsmethoden:
        \begin{itemize}
            \item - Bildaufteilung
            \item - verwendete Filter: Gauß, Dilatation, Custom-Filter
        \end{itemize}
    \end{itemize}

    \subsection{Maßnahmen zur Fehlerbehandlung}
    \begin{itemize}
        \item Becher nicht erkannt
        \item Becher fälschlicherweise erkannt
        \item nicht alle Becher in der globalen Erkennung erkannt
        \item mehr als 36 Becher in der globalen Erkennung erkannt
    \end{itemize}

    \newpage
    \section{Ergebnisse}

    \newpage
    \section{Diskussion}

    \newpage
    \section{Fazit}

    \newpage
    \section{Ausblick}

    \newpage
    \section{Danksagung}

    \newpage
    \section{Anhang}

\end{document}

